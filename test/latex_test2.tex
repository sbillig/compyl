%
%  untitled
%
%  Created by sean on .
%  Copyright (c)  __MyCompanyName__. All rights reserved.
%
\documentclass[]{article}

% Use utf-8 encoding for foreign characters
\usepackage[utf8]{inputenc}

% Setup for fullpage use
\usepackage{fullpage}

% Uncomment some of the following if you use the features
%
% Running Headers and footers
%\usepackage{fancyhdr}

% Multipart figures
%\usepackage{subfigure}

% More symbols
%\usepackage{amsmath}
%\usepackage{amssymb}
%\usepackage{latexsym}

% Surround parts of graphics with box
\usepackage{boxedminipage}

% Package for including code in the document
\usepackage{listings}

% If you want to generate a toc for each chapter (use with book)
\usepackage{minitoc}

% This is now the recommended way for checking for PDFLaTeX:
\usepackage{ifpdf}

%\newif\ifpdf
%\ifx\pdfoutput\undefined
%\pdffalse % we are not running PDFLaTeX
%\else
%\pdfoutput=1 % we are running PDFLaTeX
%\pdftrue
%\fi

\ifpdf
\usepackage[pdftex]{graphicx}
\else
\usepackage{graphicx}
\fi

\usepackage{verbatim}
\newenvironment{code}{\footnotesize\verbatim}{\endverbatim\normalsize}

\title{A LaTeX Article}
\author{  }

\date{}

\begin{document}

\ifpdf
\DeclareGraphicsExtensions{.pdf, .jpg, .tif}
\else
\DeclareGraphicsExtensions{.eps, .jpg}
\fi

\maketitle


\begin{abstract}
\end{abstract}

\section{Introduction}

This is the introduction to my fantastic Erlang package.  Here's how the package starts:

\begin{code}
	-module(latex_test2).
	-compile(export_all).
\end{code}

Since this is a ``Literate Programming'' document, we need to tell the compiler
to strip out everything that isn't code.

\begin{code}
	-compile({text_transform, literate, latex}).
\end{code}

And now we can add some functions, as per usual.

\begin{code}
	double(X) -> X*2.
\end{code}

This one should trigger an ``unused variable'' warning.
\begin{code}
	add(X,Y,Z) -> X + Y.
\end{code}






\bibliographystyle{plain}
\bibliography{}
\end{document}
